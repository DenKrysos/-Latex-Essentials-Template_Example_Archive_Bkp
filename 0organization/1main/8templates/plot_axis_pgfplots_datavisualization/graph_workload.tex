
\begin{figure}[!htpb]%
	\centering%
	\begin{tikzpicture}
		\begin{axis}[
			width=0.99\columnwidth,
			height=7cm,
			grid=major,
			legend style={at={(1,1)},anchor=north east,xshift=-0.2cm,yshift=-0.2cm},
			% title={Initialisierung und Entfernung vom AP},
 			% xtick={1,2,3},
			% x tick label style={/pgf/number format/1000 sep=},
			% scaled x ticks = false,
			xlabel={Client ID},
			scaled y ticks={base 10:-3},
			%scale ticks below exponent={-3},
            % yticklabel={$\pgfmathprintnumber{\tick}$\%},
            y tick label style={
                %/pgf/number format/1000 sep= 
                /pgf/number format/.cd,
                %fixed,
                precision=3,
                /tikz/.cd
            },
			% extra y ticks={54},
			% extra y tick labels={{456,1},{1022,4}},
			% extra y tick style={grid=major,
			% 	tick label style={xshift=-1cm}},
			ylabel={Bytes sent per Client},
			%enlarge x limits=0.01,
			max space between ticks=20,
			% try min ticks=1
			]
			%
			\addplot table [x=ach_clients,
							y=ach_sentBytes]
				{\DenKrGraphicsRootDir/graph_workload.csv};
			\addlegendentry{Active-Chain}
			%
			\addplot table [x=ac_clients,
							y=ac_sentBytes]
				{\DenKrGraphicsRootDir/graph_workload.csv};
			\addlegendentry{Active-Client}
			%
			\addplot table [x=wch_clients,
							y=wch_sentBytes]
				{\DenKrGraphicsRootDir/graph_workload.csv};
			\addlegendentry{Waiting-Chain}
		\end{axis}
	\end{tikzpicture}
	\caption{Workload Distribution.}
	\label{fig:measurement_workload}
\end{figure}%