%===================================================
% Formatting the List of Acronyms in different ways
%===================================================



%======================================
% Two quick ones. Rather simple. Appropriate for Papers / Articles
%
\usepackage[%
	acronym=true,% acronyms
% 	nomain,%
	nonumberlist,%
	nopostdot,%
	seeautonumberlist,%
	shortcuts,%
	section=chapter,%
	toc=false,%
	nogroupskip=true,%
    prefix,% makes -extra load -prefix
]{glossaries-extra}% glossaries  glossaries-extra
\setabbreviationstyle[acronym]{long-short}% To have true Gls-first output with -extra
\makeglossaries%
\loadglsentries{glossaries.tex}%
%\glsdisablehyper% Disables Hyperlinks
\renewcommand{\glossarysection}[2][]{}%Completely removes the section heading (re-defines the internally used command to just do nothing)
%
% ---------------------
% As List. (No Indent between)
\newlist{myglossarylist}{description}{10}%
\setlist[myglossarylist]{%
    align=left,%
    partopsep=0ex,%
    parsep=0ex,%
    itemsep=0ex,%
    leftmargin=0em,%
    labelsep=0.7em,%
    %itemindent=0em,%
    %labelindent=1em,%
    listparindent=\parindent,%
}%
\setlist[myglossarylist,2]{%
    align=left,%
    leftmargin=2em,%
    itemindent=-2em,%
}%
\newglossarystyle{mylist}
{% based on list style (adapt as required)
    \setglossarystyle{list}%
	\renewenvironment{theglossary}%
		{\begin{myglossarylist}}{\end{myglossarylist}}%
    \renewcommand{\glossentry}[2]{%
        \item[\glsentryitem{##1}%
            \glstarget{##1}{\glossentryname{##1}}]%
        \glossentrydesc{##1}\glspostdescription%
        \space\space##2%
    }%
}%
%
% ---------------------
% As Table. (Separate Columns)
\newglossarystyle{mytable}
{% based on tabular style
    \setglossarystyle{super}%
	% \renewenvironment{theglossary}%
	% 	{\begin{supertabular}{ll}}{\end{supertabular}}%
    \renewcommand{\glossentry}[2]{%
        \bfseries\glsentryitem{##1}%
            \glstarget{##1}{\glossentryname{##1}}&%
        \glossentrydesc{##1}\glspostdescription%
        \space\space##2%
        \\%
    }%
}%
% For the Width of the Description Field (2nd Column)
\setlength{\glsdescwidth}{0.8\hsize}% Default: 0.6\hsize



% Printing it (via selecting one of the Styles above) (somewhere towards the end)
\section*{List of Acronyms}
\addcontentsline{lof}{section}{List of acronyms}
\printglossary[type=\acronymtype,style=mytable]% style=list, index



%======================================
% More complex ones found in "0organization/1main/3settings/4customization/glossaries_style.tex"
%   in either the Boilerplate Project or "Setup_Setting_STY"






%===================================================
% Formatting the Appearance of Acronyms in the Text
%   (Is a little hacky, as it may interfer with some settings of glossaries(-extra) (like \setabbreviationstyle)
%===================================================
\usepackage{textcase,mfirstuc,xspace}%
\defglsentryfmt[acronym]{%
    \ifglsused{\glslabel}{%
        \glsgenentryfmt%
    }{%
        \MakeLowercase{\glsentrydesc{\glslabel}}\xspace(\glsentrytext{\glslabel})%
    }%
}%


